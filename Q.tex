\documentclass[12pt]{article}

\usepackage{setspace}
\doublespacing

\title{\textbf{Review of How to Share a Quantum Secret}}
\author{Haobei Song\\University of Waterloo}
\date{March 28 2017}

\begin{document}

\pagenumbering{gobble}
\maketitle
\newpage

\pagenumbering{arabic}

\begin{abstract}
	Secret sharing also known as secret splitting has been developed considerably ever since its invention in 1979. With the emerging of quantum computing, study of secret sharing protocols implemented by quantum physics becomes an active area filled with interest and intrigue due to the eccentric properties from quantum mechanics such as entanglement and interference between quantum states.
	In a $(k,n)$ threshold scheme, a secret quantum state could be transformed into n shares and kept by different participants among whom only k or more shares combined can reconstruct the secret, with another restriction that any set of shares fewer than k contain no information at all about the secret and thus cannot be used to reconstruct the secret.
	This paper then shows that it is the "no-cloning theorem" solely that places the restriction on the existence of the threshold schemes $(k, n)$ where $n \le 2k$, along with the algorithms for constructiong all of the threshold schemes belonging to this category efficiently. In addition, it also explains the reason why the shares distributed by a $(k,n)$ threshold scheme with $k \leq n \le 2k-1$ must be in a global mixed state.
	

\end{abstract}
\section{Introduction}
Since the first time the concept of secret sharing was put forward independently by Adi Shamir and George Blakley in 1979, it has gained tremendous attention and 
	Secret sharing also known as secret splitting is refered to as protocols or methods for secret dstribution among a gathering of parties each of whom is provided with a share of the secret. hao


\end{document}
